\newpage
\blankpage

\begin{center}
    \section*{Resumo}
\end{center}

\noindent 
A segmentação de tecidos de fotografias de úlceras crônicas em membros inferiores é uma abordagem não intrusiva para análises dermatológicas.
Este trabalho envolve o estudo e implementação de técnicas de processamento digital de imagens aplicadas à imagens de úlceras em membros inferiores, com o objetivo de segmentar e identificar as diversas classes de tecidos existentes nesse tipo de lesão.
Para esse propósito foram desenvolvidas uma solução \textit{ad-hoc} e uma solução baseada em aprendizado de máquina.
Essa última, denominada \system, é um método que combina estratégias de aprendizado supervisionado e não supervisionado para melhorar a segmentação das imagens dermatológicas.
Dada uma foto de úlcera capturada de acordo com um protocolo médico fixo, a primeira fase do sistema realiza uma classificação de pontos de interesse, enquanto filtros de pré-processamento são empregados para suavizar o ruído da imagem.
A imagem limpa é enviada posteriormente para uma segunda fase que consiste em divisão e conquista, baseada em um algoritmo de construção de superpixels.
O método \system divide membro inferior em regiões de interesse com bordas bem definida e agrupa os superpixels utilizando o algoritmo DBSCAN, que é baseado em similaridade.
As duas fases do \system foram avaliadas em um conjunto real de feridas dermatológicas. 
Avaliações empíricas em amostras representativas de até $ 100.000$ pontos mostram que uma rede neural artificial Perceptron Multi-Layer com o algoritmo de treinamento Levenberg-Marquardt (Coeficiente de Kappa-Cohen = $ 0,971 $, Sensibilidade = $ 0,98 $ e Especificidade = $ 0,98 $) supera outros classificadores como a primeira fase da \system.
Além disso, ensaios experimentais com o DBSCAN e cinco funções de distância ($ L_1 $, $ L_2 $, $ L_ \infty $, Canberra e BrayCurtis) indicam que a função $ L_1 $ gera menos grupos em comparação a outras funções, percebendo-se um decaimento exponencial da quantidade de grupos em função da taxa de similaridade.
O critério do ``cotovelo'' foi empregado para encontrar o limite do DBSCAN com a função $ L_1 $ como parametrização da segunda fase do sistema.
A avaliação final do \system com as duas fases devidamente configuradas em um conjunto rotulado de imagens de úlcera indica que os tecidos foram corretamente segmentados dentro de uma razão de erro absoluto médio de $0,05$, o que ilustra o impacto de eficácia do método \system para a segmentação de feridas.

\vspace{1cm}
\noindent \textbf{Palavras-chave: Imagens de feridas, Superpixels, MLP, DBSCAN }


\newpage
\blankpage

\begin{center}
    \section*{Abstract}
\end{center}

\noindent
Tissue segmentation in photographs of lower limb chronic ulcers is a non-intrusive approach that supports dermatological analyses.
This study involves the design and implementation of digital image processing techniques  applied to ulcer images in lower limbs, dining at identifying and segmenting different classes of tissues within dermatological wounds.
According, two solutions are discussed: one ad-hoc proposal, and one machine-learning-based approach, called \system.
The latter solution combines supervised and unsupervised learning strategies for enhancing the segmentation of dermatological ulcers.
Given an ulcer photo captured according to a fixed protocol, \system first phase performs a pixelwise classification of points of interest, whereas pre-processing filters are employed for the smoothing of image noise.
The cleaned image is further sent to the \system divide-and-conquer second phase.
It builds upon SLIC superpixel construction algorithm for dividing the lower limb into regions of interest with well-defined borders, and clusters the superpixels by taking advantage of the similarity-based DBSCAN algorithm.
We set up the phases of the method by using a real annotated set of dermatological wounds, and empirical evaluations on representative samples up to $100{,}000$ points showed a compact Multi-Layer Perceptron with Levenberg--Marquardt training algorithm (Cohen-Kappa Coefficient = $0.971$, Sensitivity = $0.98$, and Specificity = $0.98$) outperformed other classifiers as \system first phase.
Additionally, experimental trials on DBSCAN with five distance functions ($L_1$, $L_2$, $L_\infty$, Canberra, and BrayCurtis) indicated $L_1$ function provided fewer groups in comparison to the competitors, and the number of clusters was an exponential decay to the similarity ratio.
Accordingly, we used the ``elbow'' criterion for finding the $L_1$-based DBSCAN threshold as the \system second phase parameterization.
We evaluated the fine-tuned setting of \system over a labeled set of ulcer images, and wounded tissues were segmented within a $0.05$ Mean Absolute Error ratio.
These results illustrate the method efficacy for wound segmentation.

\vspace{1cm}
\noindent \textbf{Keywords: Wound images, Superpixels, MLP, DBSCAN}
