\section{Conclusions and Future Work}

This study has discussed a two-phase segmentation approach for detaching pieces of tissues within and around dermatological ulcers.
The method, called \system, combines supervised and unsupervised learning for reducing the number of regions of interest and the clustering of visually similar tissues.
Empirical evaluations on a real image set indicated \system effectively separated skin and background pixels at $.971$ CKC, and efficiently reduced the number of superpixels to be clustered in up to $48\%$.
Additionally, experiments showed $L_1$ was the metric that generated fewer groups in comparison to the competitors, and the number of clusters is an exponential decay to the similarity threshold.

Accordingly, \system was fine-tuned with elbow criterion for determining the clustering threshold.
We also compared \system to manual segmentation, and results showed \system quantified the pixelwise area at a $.05$ MAE ratio.
Future works include \system extension for handling superpixel convolutional features as well as the evaluation of hierarchical clustering methods as \system second phase.

