% \clearpage
\section{Conclusão}\label{sec:conclusion}

Neste trabalho foram  aplicadas técnicas de processamento de imagens digitais e aprendizado de máquina com o objetivo de desenvolver uma ferramenta CAD especializada capaz de auxiliar no diagnóstico e acompanhamento de úlceras em membros inferiores à partir de fotografias das lesões.
Ao contrário das abordagens existentes, as propostas desse trabalho consideram que a área em volta da lesão e os tecidos não lesionados são igualmente importantes para o tratamento da ulceração.
Por exemplo, os tecidos na borda da lesão podem indicar que a úlcera está cicatrizando ``rápido'' demais ou que a lesão está gerando um excesso de umidade no local.
Essas informações não são úteis para a detecção dos tipos dos tecidos lesionados em si, mas são de grande importância para o acompanhamento do procedimento terapêutico realizado pelo dermatologista ou outro profissional da saúde.

Não obstante, durante os estudos realizados foi detectado que seria desejável restringir o volume de dados a ser analisado para apenas a região de derme (lesionada ou não) para diminuir os custos computacionais associados ao processamento de imagens.
Portanto, uma primeira necessidade mais objetiva detectada foi a remoção da área do fundo da imagem (lençóis, etc.) da área de interesse (pele dos pacientes). 
Essa tarefa é particularmente desafiadora, pois deve-se remover dados inúteis, porém com a menor taxa de erro possível para que não haja interferência no resultado resultado final da segmentação.

Uma primeira solução para a remoção do fundo, foi o projeto de uma solução \textit{ad-hoc}, composta de uma conversão de imagem colorida no padrão RGB para uma imagem em tons de cinza. 
Nessa transformação, após testes para determinar a influência de cada uma das componentes de cor, os pesos das componentes cromáticas foram definidos de forma arbitrária de modo que apenas a área de interesse se tornasse visível com a aplicação de uma máscara.
A separação da imagem em duas classes (fundo e pele) foi obtida com o uso de uma técnica de binarização com limiar, calculado automaticamente pelo método de Otsu.
Como esperado para essa metodologia, a imagem resultante apresentou ruídos que foram removidos por filtros de processamento de imagens baseados em morfologia matemática. 

Embora os resultados dessa metodologia \textit{ad-hoc} tenham apresentado alguma eficiência em segmentar a pele do fundo da imagem, foi verificado que ao se fazer uso de um único limiar há a perda de trechos da imagem erroneamente identificados como fundo.
Para remover essa característica indesejada de um único ponto de corte, uma nova técnica de segmentação foi adotada -- o uso de classificadores oriundos do aprendizado de máquina.
Esses métodos permitem definir múltiplos critérios de corte que, dentro de algumas perspectivas, podem ser vistos como múltiplos limiares.
Os filtros de suavização morfológica foram mantidos ao final do processo de classificação pixel-a-pixel.\\~\\

Como método de aprendizado de máquina, foram testados seis classificadores, sendo eles: Na\"{i}ve-Bayes, Bayes-Net, \textit{k}-Nearest Neighbor, Multi-Layer Perceptron (MLP), DecisionTree, RandomTree e RandomForest. 
Desses, por meio de avaliação de qualidade baseada no coeficiente de Cohen-Kapa e curva ROC, a MLP se mostrou muito mais eficaz para classificar pixels entre pele e fundo da imagem.
Essa abordagem foi, então, adotada como a primeira fase do método final proposto nesse trabalho, o \system.
Uma vez selecionada a área de interesse, o próximo passo do \system consiste em separar regiões de pele saudável de regiões lesionadas.
Para tanto, aplica-se o método SLIC para agrupar pixels similares em superpixels com bordas bem definidas.
Esses superpixels podem ser representados em um espaço tridimensional dado pelo modelo de cores LAB, de forma que torna-se possível agrupá-los de acordo com a sua densidade no espaço multidimensional pelo algoritmo particional DBSCAN.

Portanto, o maior desafio nessa segunda fase do \system é encontrar o ajuste da parametrização adequada para o método DBSCAN.
Foram analisadas cinco funções de distâncias métricas, sendo a $L_1$ a função de distância mais ``estável'', no sentido de produzir sempre menos grupos para todas as imagens avaliadas.
Na sequência, as avaliações empíricas mostraram que o número de grupos resultantes da aplicação do método DBSCAN sobre superpixels de imagens de úlceras é uma função de decaimento exponencial em relação ao limiar de similaridade.
Com isso, foi usado o critério do ``cotovelo'' (similar ao usado para o \textit{Scree-Plot} no método de análise de componentes principais) para se fixar o valor de similaridade de $3\%$ da distância máxima no espaço unitário de cores a ser usado pelo DBSCAN.

O método \system ajustado com o classificador MLP e o método de agrupamento DBSCAN com função $L_1$ e limite de similaridade de $3\%$ foi aplicado sobre um conjunto real de úlceras em membros inferiores que já havia sido segmentado manualmente por especialistas da área.
Assim, foi possível comparar as segmentações geradas pela solução proposta nesse trabalho contra especialistas do domínio.
Para o conjunto de imagens avaliadas, o \system obteve um Erro Absoluto Médio de apenas 0,05 (variância de 0,01), o que permite concluir que as segmentações automáticas alcançaram resultados próximos aos dos especialistas humanos.

Como conclusão final, o método \system proposto empregou uma combinação de técnicas de aprendizado supervisionado e não-supervisionado capazes de segmentar regiões de úlcera em membros inferiores e fornecendo resultados e observações empíricas que podem ser usadas por trabalhos relacionados em um futuro próximo.


\subsection{Dificuldades Encontradas}

Na execução desse trabalho foram superadas diversas dificuldades, tais como:

\begin{enumerate}

    \item \textit{Seleção dos métodos}.
    Uma vez que não há indicação clara na literatura de quais métodos (ou abordagens) são os mais eficientes para o processamento de imagens de úlceras, foi necessário um estudo aprofundado das metodologias aplicáveis ao problema.

    \item \textit{Cronograma de atividades}. 
    Os temas necessários para realização da pesquisa eram ``cíclicos'', sendo necessário revisar diversos conceitos de forma paralela.

    \item \textit{Framework de implementação}. 
    Foram testadas várias plataformas de implementação e linguagens de programação, como C++, antes que fosse identificada a mais adequada ao propósito do trabalho (Matlab). 
    Portanto, algumas partes do trabalho precisaram ser adaptadas ou reescritas algumas vezes.

    \item \textit{Tempo de Execução}. 
    Alguns testes requisitaram grande esforço computacional, levando horas até que resultados pudessem ser apreciados. 
    Isso fez com que a identificação de ajustes necessários ao trabalho levassem dias para serem realizados.

    \item \textit{Biblioteca/Espaço de Estudos}. 
    O acervo da biblioteca do INFES contempla apenas livros-texto das disciplinas obrigatórias.
    Frequentemente, alguns assuntos necessários ao desenvolvimento desse trabalho não estavam disponibilizados.
    Além disso, ocorreram múltiplas greves e paralisações que afetaram o funcionamento da biblioteca.
  
\end{enumerate}

\subsection{Publicação}

Os resultados principais decorrentes desse trabalho foram publicados em uma conferência internacional de impacto nas áreas de Computação e Informática Biomédica.
O artigo foi apresentado no dia dois de junho de 2019 na Espanha, mas o D.O.I. (\textit{Digital Object Identifier}) da publicação ainda não se encontra livremente disponibilizado na internet\footnote{https://www.doi.org/}. 
A citação completa do artigo, por enquanto, é como se segue.

\begin{itemize}
    \item Silva, Wellington S.; Jasbick, Daniel L.; Wilson, Rodrigo E.; Azevedo-Marques, Paulo M.; Traina, Agma J. M.; Santos, Lucio F. D.; Jorge, Ana E. S.; Oliveira, Daniel; Bedo, Marcos V. N. \textit{A Two-Phase Learning Approach for the Segmentation of Dermatological Wounds}. In:~\textit{Proceedings of 32nd International Symposium on Computer-Based Medical Systems.} Cordoba – ESP: IEEE. v. 1, p. 1–6. 
\end{itemize}

Além dessa publicação, os resultados iniciais obtidos com o método \textit{ad-hoc} foram submetidos em relatório técnico PROAES/UFF do projeto ``\textit{Implementação de extratores de características para imagens de úlcera venosa}'' de 2017. 

\subsection{Premiação}

Os resultados desse trabalho foram reportados na forma de um relatório e um artigo científico, apresentados na V Semana de Desenvolvimento Acadêmico da UFF e no $32^o$ Congresso CBMS, respectivamente.
Ambos os trabalhos foram premiados pelas comissões técnicas avaliadoras dos eventos, como se segue.

\begin{enumerate}
    \item \textbf{(Convite para re-apresentação na Semana Nacional de Tecnologia -- INFES)} -- Implementação de extratores de característica para imagens de úlcera venosa.  \textit{V Semana de Desenvolvimento Acadêmico}.
    S. A. Pádua: Pró-Reitoria de Assuntos Estudantis, 2017

    \item \textbf{(Menção Honrosa -- Finalist Paper)} -- A Two-Phase Learning Approach for the Segmentation of Dermatological Wounds. 
    \textit{$32^o$ International Symposium on Computer-Based Medical Systems}. Cordoba/ESP, 2019.
\end{enumerate}


\subsection{Trabalhos Futuros}

O estudo e as avaliações experimentais desse trabalho serão usados como base para estender a investigação sobre imagens de úlceras dermatológicas.
Em particular, como trabalhos futuros, pretende-se investigar as seguintes vertentes derivadas dessa pesquisa:

\begin{itemize}
    
    \item \textit{Comparar o método DBSCAN com outros métodos de agrupamento como a segunda fase da solução \system}. 
    Outros métodos de agrupamento, em especial aqueles baseados em grafos, podem ser usados como a segunda fase da solução \system. 
    Determinar qual método é o mais adequado para esse domínio requer avaliação teórica e experimental dos algoritmos comparados.
    
    \item \textit{Rotular os tecidos afetados}.
    As regiões segmentadas pelo \system podem ser automaticamente rotuladas por classificadores treinados com exemplos de fibrina, necrose, tecidos de borda calosa e outros. 
     
    \item \textit{Construir um aplicativo para hospedar a solução \system}.
    O método \system pode ser incorporado diretamente em um aplicativo móvel que capture fotos de úlceras em membros inferiores e as analise automaticamente.
    
    \item \textit{Sugerir tratamentos baseados em diagnósticos similares}.
    A saída do método \system permite obter uma imagem estruturada em trechos de diferentes tipos de lesão.
    Essa estrutura pode ser usada para comparar a fotografia analisada com imagens laudadas e previamente armazenadas em um banco de dados.    
\end{itemize}
